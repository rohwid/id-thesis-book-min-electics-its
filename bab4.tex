\chapter{PENGUJIAN}
\label{chap:chap4_eval}
\vspace{1ex}

\section*{}
Inti dari BAB \ref{chap:chap4_eval} pada penelitian adalah pengujian dengan berbagai parameter untuk \textit{stats} pemain dan musuh. Seperti halnya \textit{stats} pemain dan musuh pada permainan dengan genre \textit{turn-based} dan \textit{action} RPG.
\vspace{1ex}

\section{Pengujian dalam Pembuatan \textit{Stats} Pemain pada Permainan dengan Genre \textit{Turn-based} RPG}
\label{sec:sec4_eval_turn-based_player}
\vspace{1ex}

Pada bagian ini setiap langkah yang akan diuji sudah dijelaskan pada Sub-bab \ref{sec:sec3_player_stats}, dimana pada bagian ini membahas tentang pembuatan \textit{stats} pemain untuk permaian dengan genre \textit{turn-based} RPG. Melalui berbagai proses seperti yang dijelaskan pada Sub-bab \ref{sec:sec3_player_stats}, maka pada beberapa Sub-bab dibawah ini adalah langkah-langkah dalam proses pengujiannya.
\vspace{1ex}

\subsection{Distibusi Level, HP, dan MP Pemain}
\label{sec:sub_sec4_eval_dist_hp_mp_level}
\vspace{1ex}

Seperti yang sudah dijelaskan pada Sub-bab \ref{sec:sub_sec3_enemy_hp_mp_stats}, dimana pada bagian ini membahas tentang pembuatan level, HP dan MP pemain untuk permaian dengan genre \textit{turn-based} RPG. Berikut adalah hasil dari prosess pembuatan Level, HP, dan MP yang dipeeroleh dari perhitungan pada persamaan \ref{eq:hp_player} dan \ref{eq:mp_player} dengan data masukan pada Tabel \ref{tb:player_input_variable} yang kemudian menghasilkan data seperti yang ditunjukkan pada Tabel \ref{tb:player_hp_mp}.

\begin{longtable}{|l|l|l|}
	\caption{Hasil Perhitungan HP dan MP}
	\label{tb:player_hp_mp}\\
	\hline
	\rowcolor[HTML]{C0C0C0} 
	\textbf{Levels} & \textbf{HP} & \textbf{MP} \\ \hline
	1 & 159 & 89 \\ \hline
	2 & 163 & 93 \\ \hline
	3 & 167 & 97 \\ \hline
	4 & 171 & 101 \\ \hline
	5 & 175 & 105 \\ \hline
	6 & 179 & 109 \\ \hline
	7 & 183 & 113 \\ \hline
	... & ... & ... \\ \hline
	\textbf{100} & \textbf{555} & \textbf{485} \\ \hline
\end{longtable}
\vspace{1ex}

Hasil perhitungan tersebut terlihat membentuk pola linier, yang mana nilai HP dan MP terus naik secara konstan ke atas sesuai dengan kenaikan levelnya seperti yang direpresentasikan pada Gambar \ref{fig:hp_player} dan Gambar \ref{fig:mp_player}.

\begin{figure} [!h] \centering
	\includegraphics[scale=0.45]{img/PlayerHpDistrib.png}
	\caption{Kenaikan HP setiap levelnya.}
	\label{fig:hp_player}
\end{figure}

Jika melihat Gambar \ref{fig:hp_player} dengan jumlah HP dari pemain yang terus naik mengikuti pola yang dihasilkan pada Tabel \ref{tb:player_hp_mp}, yang mana nilai tersebut diperoleh dari inisiasi variabel pada Tabel \ref{tb:player_input_variable} yaitu \textit{Max Level}, \textit{Start} HP dan \textit{Next} HP. Variabel-variabel tersebut dihitung dengan menggunakan persamaan \ref{eq:hp_player} agar membentuk pola kenaikan HP setiap levelnya seperti yang ditunjukan pada Gambar \ref{fig:hp_player}. Sama seperti pada Gambar \ref{fig:hp_player}, pada Gambar \ref{fig:mp_player} dengan jumlah MP dari pemain yang terus naik mengikuti pola yang dihasilkan pada Tabel \ref{tb:player_hp_mp}, yang mana nilai tersebut diperoleh dari inisiasi variabel pada Tabel \ref{tb:player_input_variable} yaitu \textit{Max Level}, \textit{Start} MP dan \textit{Next} MP. Kemudian variabel-variabel tersebut dihitung dengan menggunakan persamaan \ref{eq:mp_player} agar membentuk pola kenaikan MP setiap levelnya seperti yang ditunjukan pada Gambar \ref{fig:mp_player}.

\begin{figure} [!h] \centering
	\includegraphics[scale=0.45]{img/PlayerMpDistrib.png}
	\caption{Kenaikan MP setiap levelnya.}
	\label{fig:mp_player}
\end{figure}
\vspace{1ex}

\subsection{Distibusi Stats Pemain}
\label{sec:sub_sec4_eval_player_stats}
\vspace{1ex}

Pada bagian ini setiap langkah yang akan diuji sudah dijelaskan pada Sub-bab \ref{sec:sub_sec3_stat_pemain}, dimana pada bagian ini membahas tentang pembuatan \textit{stats} pemain untuk permaian dengan genre \textit{turn-based} RPG. Melalui berbagai proses seperti yang dijelaskan pada Sub-bab \ref{sec:sub_sec3_stat_pemain}, melalui persamaan \ref{eq:KNN_bayes_player_stats} dan beberapa persamaan yang digunakan sebelumnya pada Sub-bab \ref{sec:sub_sec3_stat_pemain} dihasilkan data seperti yang ditunjukan pada Tabel \ref{tb:player_battle_stats}, dan bila divisualisasikan hasilnya akan tampak seperti pada Gambar \ref{fig:stats_player}.

\begin{longtable}{|l|l|l|l|l|l|}
	\caption{Sampel hasil perhitungan dan distribusi stats dengan $k-$NN}
	\label{tb:player_battle_stats}\\
	\hline
	\rowcolor[HTML]{C0C0C0} 
	\textbf{Levels} & \textbf{Strength} & \textbf{Magic} & \textbf{Endurance} & \textbf{Speed} & \textbf{Luck} \\ \hline
	1 & 1 & 2 & 0 & 0 & 1 \\ \hline
	2 & 0 & 2 & 0 & 2 & 0 \\ \hline
	3 & 1 & 0 & 0 & 0 & 0 \\ \hline
	4 & 1 & 1 & 0 & 0 & 0 \\ \hline
	5 & 1 & 1 & 0 & 0 & 0 \\ \hline
	6 & 1 & 1 & 0 & 2 & 1 \\ \hline
	7 & 0 & 0 & 0 & 0 & 1 \\ \hline
	8 & 1 & 0 & 0 & 0 & 0 \\ \hline
	... & ... & ... & ... & ... & ... \\ \hline
	\textbf{100} & \textbf{1} & \textbf{0} & \textbf{0} & \textbf{0} & \textbf{0} \\ \hline
\end{longtable}

\begin{figure} [!h] \centering
	\includegraphics[scale=0.50]{img/PlayerStatsDistrib.png}
	\caption{Kenaikan stats pemain setiap levelnya.}
	\label{fig:stats_player}
\end{figure}

Pada Tabel \ref{tb:player_battle_stats} adalah data \textit{stats} dari pemain yang dihasilakn melalui penambahan nilai pada \textit{stats} secara acak pada setiap \textit{stats}. Seperti yang dijelaskan pada persaamaan \ref{eq:nbayes_class}, \ref{eq:KNN_distance_stats}, dan persamaan \ref{eq:KNN_bayes_player_stats} saat nilai \textit{stats} di tambahkan secara acak dari 1 sampai dengan 2 pada level yang berjarak antara 1 sampai dengan 100. Selanjutnya representasi dari hasil penambahan \textit{stats} tersebut yang ditunjukan melalui Gambar \ref{fig:stats_player} yang mana nilai dari setiap \textit{stats} terus naik sesuai dengan level pemain yang juga terus naik. 
\vspace{1ex}

\section{Pengujian dalam Pembuatan \textit{Stats} Musuh pada Permainan dengan Genre \textit{Turn-based} RPG}
\label{sec:sec4_eval_turn-based_enemy}
\vspace{1ex}

Pada bagian ini setiap langkah yang akan diuji sudah dijelaskan pada Bagian \ref{sec:sec3_enemy_stats}, dimana pada bagian ini membahas tentang pembuatan \textit{stats} musuh untuk permaian dengan genre \textit{turn-based} RPG. Melalui berbagai proses seperti yang dijelaskan pada Bagian \ref{sec:sec3_player_stats}, maka pada beberapa Sub-bab dibawah ini adalah langkah-langkah dalam proses pengujiannya.
\vspace{1ex}

\subsection{Distibusi Level Musuh}
\label{sec:sub_sec4_eval_dist_enemy_level}
\vspace{1ex}

Seperti yang sudah dijelaskan pada Sub-bab \ref{sec:sub_sec3_enemy_level}, dimana pada bagian ini membahas tentang pendistribusian level musuh untuk permaian dengan genre \textit{turn-based} RPG. Berikut adalah hasil dari proses distribusi level yang diperoleh dari perhitungan pada persamaan \ref{eq:enemy_levels1}, \ref{eq:enemy_levels2}, \ref{eq:sub_enemy_levels1}, \ref{eq:sub_enemy_levels2} dan persamaan \ref{eq:probability_enemy_levels}. Setelah melalui tahapan tersebut dan dengan variabel masukan pada Tabel \ref{tb:enemy_input_variable} maka level yang dihasilkan akan seperti pada Tabel \ref{tb:enemy_level_distrib}.

\begin{longtable}{|l|l|l|}
	\caption{Hasil level yang dibuat untuk musuh.}
	\label{tb:enemy_level_distrib}\\
	\hline
	\rowcolor[HTML]{C0C0C0} 
	\textbf{No.} & \textbf{Name} & \textbf{Levels} \\ \hline
	1 & Enemy 1 & 1 \\ \hline
	2 & Enemy 2 & 1 \\ \hline
	3 & Enemy 3 & 1 \\ \hline
	4 & Enemy 4 & 1 \\ \hline
	5 & Enemy 5 & 2 \\ \hline
	6 & Enemy 6 & 2 \\ \hline
	7 & Enemy 7 & 2 \\ \hline
	8 & Enemy 8 & 2 \\ \hline
	9 & Enemy 9 & 2 \\ \hline
	10 & Enemy 10 & 2 \\ \hline
	11 & Enemy 11 & 2 \\ \hline
	12 & Enemy 12 & 2 \\ \hline
	13 & Enemy 13 & 2 \\ \hline
	14 & Enemy 14 & 3 \\ \hline
	15 & Enemy 15 & 3 \\ \hline
	... & ... & ... \\ \hline
	\textbf{400} & \textbf{Enemy 400} & \textbf{78} \\ \hline
\end{longtable}

Pada Tabel \ref{tb:enemy_level_distrib} adalah sebagian data dari level yang dihasilkan oleh program, untuk hasil lebih lengkapnya bisa dilihat pada Bagian \nameref{chap:chap6_attachment} pada Tabel \ref{tb:enemies_all_stats_1} sampai dengan Tabel \ref{tb:enemies_all_stats_15} di kolom \textit{Levels}. Kemudian persebaran level yang dihasilkan tersebut bila divisualisasikan maka hasilnya akan seperti yang ditujukkkan pada Gambar \ref{fig:enemy_level_distrib}. Grafik atau histogram yang ditunjukan pada Gambar \ref{fig:enemy_level_distrib} tersebut sangatlah tidak merata, hal tersebut dikarenakan proses penentuan level yang ditentukan secara acak.

\begin{figure} [!h] \centering
	\includegraphics[scale=0.48]{img/EnemyLevelDistrib.png}
	\caption{Distribusi Level Musuh.}
	\label{fig:enemy_level_distrib}
\end{figure}

Dilanjutkan dengan validasi dari keseimbangan persebaran level musuh, hal ini dilakukan dengan menggunakkan persamaan melalui beberapa langkah seperti yang ditunjukan oleh persamaan \ref{eq:mean_enemy_levels}, \ref{eq:varian_enemy_levels}, \ref{eq:stdev_enemy_levels} dan persamaan \ref{eq:PDF_enemy_levels}. Konsep tersebut beracuan pada Sub-bab \ref{sec:sub_sec2_gauss_bayes} tentang \textit{Gaussian Naive bayes}, dengan harapan apakah setiap data yang dihasilkan sebelumnya sudah terdistribusi dengan normal atau tidak. Jika memang benar data hasil program ini valid maka ketika musuh bertambah maka pola dari level dan jumlah musuh masih akan membentuk pola distribusi normal seperti pada Gambar \ref{fig:enemy_level_distrib_ndist}.

\begin{figure} [!h] \centering
	\includegraphics[scale=0.45]{img/EnemyLevelDistribNdist.png}
	\caption{Distribusi level musuh dalam bentuk distribusi normal.}
	\label{fig:enemy_level_distrib_ndist}
\end{figure}

Pada Gambar \ref{fig:enemy_level_distrib_ndist} jika dilihat persebaran datanya dari kiri ke kanan menggambarkan tingkat kesulitan musuh berdasarkan level, jumlah musuh pada sisi kiri berjumlah sedikit semakin ke kanan jumlahnya meningkat dan semakin ke kanan jumlah musuh kembali menjadi sedikit jumlahnya. Hal tersebut menggambarkan tingkat keseimbangan dari musuh yang dibuat, yang mana jumlah musuh dengan level menengah berjumlah paling banyak dan jumlah musuh yang sangat sulit dan sangat mudah dikalahkan berjummlah sedikit. Tujuan utama dari kondisi tersebut adalah terjadinya keseimbangan saat terjadinya pertarungan antara pemain dan musuh.
\vspace{1ex}

\subsection{Distibusi Type Musuh}
\label{sec:sub_sec4_eval_dist_enemy_type}
\vspace{1ex}

Seperti yang sudah dijelaskan pada Sub-bab \ref{sec:sub_sec3_enemy_type}, dimana pada bagian ini membahas tentang pendistribusian tipe musuh untuk permaian dengan genre \textit{turn-based} RPG. Berikut adalah hasil dari proses distribusi tipe musuh yang diperoleh dari perhitungan pada persamaan \ref{eq:enemy_types_percentage}, \ref{eq:enemy_types_dist_level}, \ref{eq:enemy_types_rest_dist_level}, \ref{eq:enemy_types_probability}, \ref{eq:enemy_rest_types} dan persamaan \ref{eq:enemy_types_rest_probability}. Setelah melalui tahapan tersebut dengan variabel masukan pada Tabel \ref{tb:enemy_input_variable} maka hasil distribusi dan penjelasan level yang dilakukan oleh beberapa persamaan tersebut akan menjadi seperti Tabel \ref{tb:enemy_type_distrib}.

\begin{longtable}{|l|l|l|l|}
	\caption{Hasil level yang dibuat untuk musuh.}
	\label{tb:enemy_type_distrib}\\
	\hline
	\rowcolor[HTML]{C0C0C0} 
	\textbf{No.} & \textbf{Name} & \textbf{Levels} & \textbf{Type} \\ \hline
	1 & Enemy 1 & 1 & 0 \\ \hline
	2 & Enemy 2 & 1 & 0 \\ \hline
	3 & Enemy 3 & 1 & 2 \\ \hline
	4 & Enemy 4 & 1 & 1 \\ \hline
	5 & Enemy 5 & 2 & 2 \\ \hline
	6 & Enemy 6 & 2 & 1 \\ \hline
	7 & Enemy 7 & 2 & 2 \\ \hline
	8 & Enemy 8 & 2 & 0 \\ \hline
	9 & Enemy 9 & 2 & 2 \\ \hline
	10 & Enemy 10 & 2 & 2 \\ \hline
	11 & Enemy 11 & 2 & 4 \\ \hline
	12 & Enemy 12 & 2 & 0 \\ \hline
	13 & Enemy 13 & 2 & 0 \\ \hline
	14 & Enemy 14 & 3 & 4 \\ \hline
	15 & Enemy 15 & 3 & 4 \\ \hline
	16 & Enemy 16 & 3 & 2 \\ \hline
	... & ... & ... & ... \\ \hline
	\textbf{400} & \textbf{Enemy 400} & \textbf{78} & \textbf{3} \\ \hline
\end{longtable}

Dari data yang tedapat pada Tabel \ref{tb:enemy_type_distrib} hanya sebagian data saja, lebih lengkapnya dapat dilihat langsung pada Bagian \nameref{chap:chap6_attachment} dalam Tabel \ref{tb:enemies_all_stats_1} sampai dengan Tabel \ref{tb:enemies_all_stats_15} di kolom \textit{Type}. Pada Tabel \ref{tb:enemy_type_distrib} dan tabel lain yang memuat tentang \textit{Type} yang berisi data $ET_{i}$ atau jenis musuh yang diisi dengan angka 0 sampai dengan 4, maksud dari angka tersebut adalah untuk mewakili indeks dari variabel \textit{Enemy Type} pada Tabel \ref{tb:enemy_input_variable}. Secara berurutan dari 0 sampai dengan 4 adalah \textit{Mixed}, \textit{Hard Magic}, \textit{Soft Magic}, \textit{Hard Strength}, dan \textit{Soft Magic}. Seluruh data tersebut jika divisualisasikan maka hasilnya seperti Gambar \ref{fig:enemy_type_distrib} sesuai yang ditargetkan oleh variabel \textit{Distribute Percentage} pada Tabel \ref{tb:enemy_input_variable} atau variabel $DP$ pada persamaan \ref{eq:enemy_types_percentage}.

\begin{figure} [!h] \centering
	\includegraphics[scale=0.48]{img/EnemyTypeDistrib.png}
	\caption{Distribusi tipe musuh.}
	\label{fig:enemy_type_distrib}
\end{figure}

Kemudian pada Gambar \ref{fig:enemy_type_distrib} adalah hasil dari proses pembuatan dan pengelompokan musuh berdasarkan tipe yang sudah sesuai dengan variabel masukan pada Tabel \ref{tb:enemy_input_variable}. Dengan musuh bertipe \textit{Mixed} berjumlah yang paling banyak, diikuti \textit{soft magic} dan \textit{soft strength} dengan harapan bahwa ini adalah musuh yang memiliki karakter \textit{magic} dan \textit{strength} namun masih mudah dikalahkan, dilanjutkan dengan yang paling sedikit adalah \textit{hard magic} dan \textit{hard strength} dengan harapan menjadi musuh berkarakter strength dan magic yang sulit dikalahkan.
\vspace{1ex}

\section{Pengujian dalam Pembuatan \textit{Stas} Pemain untuk Permainan dengan Genre \textit{Action} RPG}
\label{sec:sec4_eval_action_player}
\vspace{1ex}

Sama seperti pengujian yang dilakukan pada Sub-bab \ref{sec:sec4_eval_turn-based_player} yang mana langkahnya sudah dijelaskan pada Sub-bab \ref{sec:sec3_player_stats}, hanya saja pada bagian ini membahas tentang pembuatan dan penyesuaian \textit{stats} pada permaian untuk genre \textit{action} RPG. Melalui berbagai proses seperti yang dijelaskan pada Sub-bab \ref{sec:sec4_eval_turn-based_player} tentang \textit{turn-based} RPG, hanya saja pada bagian ini tidak memerlukan elemen yang menggambarkan kelemahan dari karakter pemain yang mana hal inilah yang disebut dengan penyesuaian. Melalui beberapa Sub-bab dibawah ini adalah langkah-langkah dalam proses pengujiannya.
\vspace{1ex}

\section{Pengujian dalam Pembuatan \textit{Stas} Musuh untuk Permainan dengan Genre \textit{Action} RPG}
\label{sec:sec4_eval_action_enemy}
\vspace{1ex}

Sama seperti pengujian yang dilakukan pada Sub-bab \ref{sec:sec4_eval_turn-based_player} yang mana langkahnya sudah dijelaskan pada Sub-bab \ref{sec:sec3_player_stats}, hanya saja pada bagian ini membahas tentang pembuatan dan penyesuaian \textit{stats} pada permaian untuk genre \textit{action} RPG. Melalui berbagai proses seperti yang dijelaskan pada Sub-bab \ref{sec:sec4_eval_turn-based_player} tentang \textit{turn-based} RPG, hanya saja pada bagian ini tidak memerlukan elemen yang menggambarkan kelemahan dari karakter pemain yang mana hal inilah yang disebut dengan penyesuaian. Melalui beberapa Sub-bab dibawah ini adalah langkah-langkah dalam proses pengujiannya.