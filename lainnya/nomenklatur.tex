\begin{center}
\large\textbf{NOMENKLATUR}
\end{center}
\vspace{1ex}
\begin{conditions}
	x &	Data x atau titik koordinat data.\\
	x_{0} & Titik koordinat data ke 0.\\
	x_{i} &	Data $x$ ke $i$.\\
	x_{n} & Data $x$ ke $n$.\\
	\theta & Data $\theta$.\\
	\theta_{i} & Data $\theta$ ke $i$.\\
	min\ d & Jarak minimum\\
	i & Urutan atau iterasi.\\
	Y & Titik Y.\\
	y & Bagian dari Y dengan banyak data.\\
	X & Titik X.\\
	D & Distance atau jarak.\\
	p & Data p atau titik koordinat data.\\
	p_{i} & Titik koordinat data ke $i$.\\
	y_{pred} & Hasil prediksi dari titik koordinat yang dicari.\\
	y_{i} & Titik koordinat yang dicari oleh $y_{pred}$.\\
	k & Nilai $k$ dalam $k$-NN.\\
	P(h|d) & Probabilitas hipotesis $h$ berdasrkan data $d$. Hal ini disebut probabilitas posterior.\\
	P(d|h) & Probabilitas dari data $d$ berdasarkan hipotesis $h$ yang bernilai benar.\\
	P(h) & Probabilitas hipotesis $h$ yang bernilai benar terlepas dari data secara keseluruhan.\\
	P(d) & Probabilitas data secara keseluruhan terlepas dari hipotesis.\\
	MAP(h) & Maximum a Posteriori Probability.\\
	P(C) & \textit{Class pobability}.\\
	C & \textit{Class} untuk dicari probabilitasnya.\\
	W & \textit{Weather} atau cuaca pada contoh kasus Naive Bayes.\\
	n & Jumlah data.\\
	\bar{x} & Rata-rata nilai $x$.\\
	\sigma(x) & Standar deviasi.\\
	PDF & Probability Density Function.\\
	B_{total} & Jumlah waktu melawan bos secara keseluruhan.\\
	HP & Health Point.\\
	MP & Magic Point.\\
	MaxSt & Jarak maksimum stats.\\
	St_{i} & Stats ke $i$.\
\end{conditions}
\newpage