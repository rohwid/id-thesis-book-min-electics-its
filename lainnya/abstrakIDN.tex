\begin{spacing}{1}
	\begin{center}
		\large\textbf{PERHITUNGAN KENAIKAN ATRIBUT \textit{GAMEPLAY} UNTUK PEMAIN DAN \textit{NON-PLAYER CHARACTER} PADA PERMAINAN \textit{ROLE-PLAYING GAME} BERBASIS K-NN DAN NAIVE BAYES}
	\end{center}
	\vspace{2ex}
	
	\begin{adjustwidth}{-0.2cm}{}
		\begin{tabular}{lcp{0.7\linewidth}}
			Nama Mahasiswa &:& Nur Rohman Widiyanto \\
			NRP &:&	07111650050205 \\
			Pembimbing &:& 1. Prof. Dr. Ir. Mauridhi Hery Purnomo, M.Eng. \\
			& & 2. Dr. Supeno Mardi Susiki Nugroho, ST., MT. \\
		\end{tabular}
	\end{adjustwidth}
	\vspace{2ex}
	
	\begin{center}
		\large\textbf{ABSTRAK}
	\end{center}
	\vspace{1ex}
	
	Permainan dengan genre \textit{Role Playing Game} (RPG) merupakan permainan yang bersifat kompetitif, antara pemain melawan pemain ataupun melawan musuh yang berupa \textit{Non-Player Character} (NPC). Banyak pengembang permainan dalam pembuatan permainan itu sendiri masih menggunakan cara manual dalam penentuan atribut \textit{gameplay} untuk karakter pemain ataupun musuh. Terlebih lagi, saat sebuah permainan memiliki banyak karakter, seperti hanyalnya banyak karakter pemain (contohnya pada JRPG dan TRPG) dan juga banyak musuh. Pada penelitian ini diimplementasikan beberapa pendekatan seperti halnya $k-$NN, Distribusi Normal, dan Naive Bayes yang akan digunakan dalam program penghitung kenaikan atribut \textit{gameplay} pada karakter pemain dan musuh secara otomatis. Pada program tersebut dibutuhkan parameter masukan yang akan menentukan atribut \textit{gameplay} yang akan dihasilkan. Untuk karakter pemain, dibuatlah skenario atribut \textit{gameplay} berdasarkan peran setiap karakter yang ingin dibuat seperti \textit{knight}, \textit{priest}, \textit{assassin}, dan lain-lain. Sedangkan pada karakter musuh, dalam distribusi atribut \textit{gameplay} juga dibagi menjadi beberapa tipe musuh yang dicontohkan seperti \textit{mixed}, \textit{hard strength}, \textit{hard magic}, dan lain-lain. Setelah itu, hasil atribut \textit{gameplay} diklasifikasi dengan menggunakan \textit{Neural Network Multiclass Classification}. Hal tersebut bertujuan untuk menghitung tingkat kesesuaian atribut \textit{gameplay} dari karakter pemain dan musuh yang dihasilkan. Operasi tersebut dilakukan secara terpisah pada atribut \textit{gameplay} pemain dan musuh, karena tidak adanya hubungan saat pembuatan atau perhitungan. Masing-masing dibagi kedalam data \textit{training} dan \textit{testing}, dengan perbandingan 70\% dan 30\%. Proses tersebut menghasilkan keluaran berupa tipe karakter pemain dan musuh pada data \textit{testing}, hal tersebut diperoleh dari proses \textit{training}. Presentase kesesuaian tersebut diperoleh dengan membandingkan tipe pada karakter yang diperoleh dari hasil klasifikasi pada data \textit{testing} dengan tipe dari karakter yang sebenarnya, maka diperolehlah sebuah presentase banyaknya karakter yang berhasil terklasifikasi.
	\vspace{2ex}
	
	% Kata Kunci : Kecerdasan buatan, \textit{Reinforcement Learning}, \textit{Non-Playable Character}, \textit{Turn-based}, RPG (\textit{Role Playing Game}).

	Kata Kunci: \textit{Role-Playing Game}, Atribut \textit{Gameplay}, $k-$NN, Naive Bayes, \textit{Neural Network}, Klasifikasi.
\end{spacing}