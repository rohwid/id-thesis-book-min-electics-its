\begin{spacing}{1}
	\begin{center}
		\large\textbf{PERHITUNGAN KENAIKAN \textit{STATS} LEVEL PEMAIN DAN \textit{NON-PLAYER CHARACTER} PADA PERMAINAN \textit{ROLE-PLAYING GAME} BERBASIS K-NN DAN NAIVE BAYES}
	\end{center}
	\vspace{2ex}
	
	\begin{adjustwidth}{-0.2cm}{}
		\begin{tabular}{lcp{0.7\linewidth}}
			Nama Mahasiswa &:& Nur Rohman Widiyanto \\
			NRP &:&	07111650050205 \\
			Pembimbing &:& 1. Prof. Dr. Ir. Mauridhi Hery Purnomo, M.Eng. \\
			& & 2. Dr. Supeno Mardi Susiki Nugroho, ST., MT. \\
		\end{tabular}
	\end{adjustwidth}
	\vspace{2ex}
	
	\begin{center}
		\large\textbf{ABSTRAK}
	\end{center}
	\vspace{1ex}
	
	Permainan dengan genre \textit{Role Playing Game} (RPG) merupakan permainan yang bersifat kompetitif, antara pemain melawan pemain ataupun melawan musuh yang berupa \textit{non-player character} (NPC). Banyak pengembang permainan dalam pembuatan permainan itu sendiri masih menggunakan cara manual dalam penentuan data statistik dari karakter pemain ataupun musuh. Bila pada sebuah permainan memiliki banyak karakter, seperti hanyalnya banyak karakter pemain (contohnya pada JRPG dan TRPG) dan banyak musuh. Pada penelitian ini diimplementasikan beberapa pendekatan seperti halnya $k-$NN (\textit{Nearest Neighbor}), Distribusi Normal dan Naive Bayes yang akan digunakan dalam membatu dalam menghasilkan data statistik pada karakter dan musuh secara otomatis. Setelah itu juga dilakukan validasi dari \textit{stats} yang dihasilkan dengan menggunakan \textit{Neural Network Multiclass Classification}. Dicoba juga penggunaan \textit{Neural Network} untuk klasifikasi \textit{hero} pada permainan Dota 2 yang kemudian mencapai 64\% \textit{hero} berhasil terklasifikasi pada data \textit{testing}. Sedangkan untuk klasifikasi karakter pemain yang dibuat menggunakan metode $k$-NN, dan Naive Bayes dengan jumlah delapan karakter mampu mencapi 50\% pada data \textit{testing}, kemudian untuk klasifikasi karakter musuh yang juga dibuat menggunakan metode tersebut dengan jumlah 400 karakter mampu mencapai 42.5\% pada data \textit{testing}.
	\vspace{2ex}
	
	% Kata Kunci : Kecerdasan buatan, \textit{Reinforcement Learning}, \textit{Non-Playable Character}, \textit{Turn-based}, RPG (\textit{Role Playing Game}).

	Kata Kunci: \textit{Role-Playing Game} (RPG), \textit{Stats}, $k-$NN, Naive Bayes, \textit{Neural Network}, Klasifikasi.
\end{spacing}