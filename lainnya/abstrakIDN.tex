\begin{spacing}{1}
	\begin{center}
		\Large\textbf{Validasi Pengaturan \textit{Stats} Otomatis pada Permainan \textit{Action} dan \textit{Turn-Based Role-Playing Games} (RPG) Berbasis K-NN dan Naive Bayes dengan \textit{Deep Learning Multiclass Classification}}
	\end{center}
	\vspace{2ex}
	
	\begin{adjustwidth}{-0.2cm}{}
		\begin{tabular}{lcp{0.7\linewidth}}
			Nama Mahasiswa &:& Nur Rohman Widiyanto \\
			NRP &:&	07111650050205 \\
			Pembimbing &:& 1. Prof. Dr. Ir. Mauridhi Hery Purnomo, M.Eng. \\
			& & 2. Dr. Supeno Mardi Susiki Nugroho, ST., MT. \\
		\end{tabular}
	\end{adjustwidth}
	\vspace{2ex}
	
	\begin{center}
		\Large\textbf{ABSTRAK}
	\end{center}
	\vspace{1ex}
	
	Permainan dengan genre \textit{action} atau \textit{turn-based} RPG (\textit{Role Playing Game}) merupakan permainan yang bersifat kompetitif, antara pemain melawan pemain ataupun melawan musuh yang berupa \textit{non-player character} (NPC). Banyak pengembang permainan dalam pembuatan permainan itu sendiri masih menggunakan cara manual dalam penentuan data statistik dari karakter pemain ataupun musuh. Bila pada sebuah permainan memiliki banyak karakter, seperti hanyalnya banyak karakter pemain (contohnya pada \textit{turn-based} RPG) dan banyak musuh. Pada penelitian ini akan diimplementasikan beberapa pendekatan seperti halnya $k-$NN (\textit{Nearest Neighbor}), Distribusi Normal dan \textit{Naive Bayes} yang akan digunakan dalam membatu dalam menghasilkan data statistik pada karakter dan musuh secara otomatis.
	\vspace{2ex}
	
	% Kata Kunci : Kecerdasan buatan, \textit{Reinforcement Learning}, \textit{Non-Playable Character}, \textit{Turn-based}, RPG (\textit{Role Playing Game}).

	Kata Kunci : \textit{Stats}, $k-$NN, Naive Bayes, Deep Learning, Klasifikasi, \textit{Role-Playing Game} (RPG).
\end{spacing}