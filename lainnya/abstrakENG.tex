\begin{spacing}{1}
	\begin{center}
		\large\textbf{THE CALCULATION OF PLAYER'S AND NON-PLAYER CHARACTER'S LEVELS STATS GROWTH IN ROLE-PLAYING GAME WITH K-NN AND NAIVE BAYES}
	\end{center}
	\vspace{2ex}
	
	\begin{adjustwidth}{-0.2cm}{}
		\begin{tabular}{lcp{0.7\linewidth}}
			By &:& Nur Rohman Widiyanto \\
			Student Identity Number &:&	07111650050205 \\
			Supervisors &:& 1. Prof. Dr. Ir. Mauridhi Hery. P, M.Eng. \\
			& & 2. Dr. Supeno Mardi Susiki. N, ST., MT. \\
		\end{tabular}
	\end{adjustwidth}
	\vspace{2ex}
	
	\begin{center}
		\large\textbf{ABSTRACT}
	\end{center}
	\vspace{1ex}
	
	The game with the Role Playing Game (RPG) genre was a competitive game, between players against other players or against enemies with form of Non-Player Character (NPC). Many game developers when makes the game itself still uses manual approach when determine the statistical data for the player or enemy character. When in a game has a lot of characters, such as only a lot of player characters (for example on JRPG and TRPG) and many enemies. In this research several approaches such as $k-$NN (Nearest Neighbor) was implemented, Normal Distribution and Naive Bayes which used to generate statistical data for player characters and enemy automatically. Then, the Neural Network Multiclass Classification will be used for validate the stats that was generated before. The Neural Network also used for the classification of heroes in the Dota 2 game which reached 64\% of heroes was successfully classified in the testing data. The player character classification which created using the $k$-NN and Naive Bayes method with the total of eight characters was able to reached 50\% in the testing data, then for the classification of enemy characters that was also made using the same method with the total of 400 characters was able to reached 42.5\% in testing data.
	\vspace{2ex}

	Keywords: Role-Playing Game (RPG), Stats, $k-$NN, Naive Bayes, Neural Network, Classification.
\end{spacing}