\begin{spacing}{1}
	\begin{center}
		\large\textbf{THE CALCULATION OF PLAYER'S AND NON-PLAYER CHARACTER'S GAMEPLAY ATTRIBUTE GROWTH IN ROLE-PLAYING GAME WITH K-NN AND NAIVE BAYES}
	\end{center}
	\vspace{2ex}
	
	\begin{adjustwidth}{-0.2cm}{}
		\begin{tabular}{lcp{0.7\linewidth}}
			By &:& Nur Rohman Widiyanto \\
			Student Identity Number &:&	07111650050205 \\
			Supervisors &:& 1. Prof. Dr. Ir. Mauridhi Hery. P, M.Eng. \\
			& & 2. Dr. Supeno Mardi Susiki. N, ST., MT. \\
		\end{tabular}
	\end{adjustwidth}
	\vspace{2ex}
	
	\begin{center}
		\large\textbf{ABSTRACT}
	\end{center}
	\vspace{1ex}
	
	The game with the Role Playing Game (RPG) genre was a competitive game, between players against other players or enemies in the form of a Non-Player Character (NPC). Many game developers in making the game itself still use manual methods in determining gameplay attributes for player or enemy characters. Especially, when the game had many player characters (for example in JRPG and TRPG) and also the enemy. In this research, several approaches were implemented such as $k$-NN, Normal Distribution, and Naive Bayes which will be used in the program to automatically calculate the growth in gameplay attributes of the player and enemy characters. The program requires input parameters that will determine the result of gameplay attributes. For the player character, the gameplay attribute scenario was created based on the role of each character that user want to make such as knight, priest, assassin, and others. Whereas for enemy characters, the distribution of gameplay attributes also divided into several types of enemies that were exemplified, such as mixed, hard strength, hard magic, and others. After that, the results of the gameplay attributes classified using the Neural Network Multiclass Classification. It aims to calculate the level of suitability of the gameplay attributes of the resulting player and enemy characters. The operation is carried out separately on the player's and enemy's gameplay attributes because there is no correlation during creation or calculation. Each result divided into training and testing data, with a ratio of 70\% and 30\%. This process produces output in the form of player's and enemy's character types on the testing data, that obtained from the training process. The percentage of suitability obtained by comparing the types of characters from the classification results in the testing data with the types of the actual characters, so the percentage of the number of characters that were classified had obtained.
	\vspace{2ex}

	Keywords: Role-Playing Game, Gameplay Attributes, $k-$NN, Naive Bayes, Neural Network, Classification.
\end{spacing}