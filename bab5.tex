\chapter{PENUTUP}
\label{sec:chap5_tutup}
\vspace{1ex}

\section*{}
Setelah penerapan metode terhadap masalah yang ingin diselesaikan pada Bab \ref{chap:chap3_metodologi} dan dilakukan pengujian terhadap metode tersebut pada Bab \ref*{chap:chap4_eval} maka didapatlah kesimpulan yang akan dijabarkan pada Sub-bab berikut. Kemudian dilanjutkan dengan saran untuk penelitian kedepannya pada Sub-bab selanjutnya.
\vspace{1ex}

\section{Kesimpulan}
\label{sec:sec4_kesimpulan}
\vspace{1ex}

Pada saat pengerjaan metodologi pada BAB \ref{chap:chap3_metodologi} yang dilanjutkan dengan pengujian yang tertuang pada BAB Bab \ref*{chap:chap4_eval}, mengharuskan untuk membagi permainan dengan genre RPG ke dalam dua golongan lagi yaitu \textit{single-character} dan \textit{multi-character}. Hal ini dilakukan demi memudahkan pembuatan program pembuatan \textit{stats}, \textit{single-character} biasanya digunakan pada permainan RPG dengan \textit{sub-genre} WRPG, ARPG, SRPG, dan MMORPG, sedangkan \textit{multi-character} biasanya digunakan pada permainan RPG dengan \textit{sub-genre} TRPG dan JRPG.
\vspace{1ex}

Di gunakannya metode $k$-NN, Distribusi Normal, dan Naive Bayes dalam melakukan proses distribusi stats dalam permainan RPG menjadikan stats dari pemain dan musuh menjadi terpola dari level terendah ke level tertinggi untuk pemain, mejadi tersebar dengan merata untuk distribusi musuh yang harus dihadapi oleh pemain disetiap levelnya. Hal semacam ini tentu saja sangat memudahkan pengembang permainan dalam mendesain permainan, terlebih lagi dalam pembuatan permainan dengan level yang tinggi, karakter, dan musuh yang berjumlah banyak.
\vspace{1ex}

\section{Saran}
\label{sec:sec4_saran}
\vspace{1ex}

Dalam penelitian kedepannya sangat disarankan untuk meneliti tentang bagaimana mensimulasikan pertaarungan secara otomatis dengan jenis \textit{Action} dan \textit{Turn-based} RPG. Hal tersebut bertujuan guna mencoba mensimulasikan nilai yang dihasilkan dari penelitian ini yang berupa \textit{stats}.