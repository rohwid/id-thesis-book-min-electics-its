\chapter{PENUTUP}
\label{sec:chap5_tutup}
\vspace{1ex}

\section*{}
Setelah penerapan metode terhadap masalah yang ingin diselesaikan pada Bab \ref{sec:chap3_metodologi} dan dilakukan pengujian dari metode tersebut pada Bab \ref*{sec:chap4_pengujian} maka didapatlah kesimpulan yang akan dijabarkan pada Sub-bab berikut. Kemudian dilanjutkan dengan saran untuk penelitian kedepannya pada Sub-bab selanjutnya.
\vspace{1ex}

\section{Kesimpulan}
\label{sec:sec4_kesimpulan}
\vspace{1ex}

Dengan mengunakan metode $k$-NN dalam melakukan proses ditribusi, khususnya pada stats dalam permainan \textit{turn-based} dan \textit{action} RPG. Maka stats dapat terdistribusi dengan baik. Hanya saja belum dicoba pada permainan yang sesungguhnya. Selain itu hal semacam ini juga sangat mempermudah dalam desain permainan, terlebih lagi untuk mendesain permainan dengan musuh dan karakter yang sangat banyak.

\section{Saran}
\label{sec:sec4_saran}
\vspace{1ex}

Dalam penelitian kedepannya sangat disarankan untuk meneliti tentang bagaimana mensimulasikan pertaarungan secara otomatis dengan jenis \textit{Action} dan \textit{Turn-based} RPG. Hal tersebut bertujuan guna mencoba mensimulasikan nilai yang dihasilkan dari penelitian ini yang berupa \textit{stats}.