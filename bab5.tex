\chapter{PENUTUP}
\label{sec:chap5_tutup}
\vspace{1ex}

\section*{}
Setelah penerapan metode terhadap masalah yang ingin diselesaikan pada Bab \ref{chap:chap3_metodologi} dan dilakukan pengujian terhadap metode tersebut pada Bab \ref*{chap:chap4_eval} maka didapatlah kesimpulan yang akan dijabarkan pada Sub-bab berikut. Kemudian dilanjutkan dengan saran untuk penelitian kedepannya pada Sub-bab selanjutnya.
\vspace{1ex}

\section{Kesimpulan}
\label{sec:sec4_kesimpulan}
\vspace{1ex}

Pada saat pengerjaan metodologi pada BAB \ref{chap:chap3_metodologi} yang dilanjutkan dengan pengujian yang tertuang pada BAB \ref{chap:chap4_eval}, mengharuskan untuk membagi permainan dengan genre RPG ke dalam dua golongan lagi yaitu \textit{single-character} dan \textit{multi-character}. Hal ini dilakukan demi memudahkan pembuatan \textit{stats}, \textit{single-character} biasanya digunakan pada permainan RPG dengan \textit{sub-genre} WRPG, ARPG, SRPG, dan MMORPG, sedangkan \textit{multi-character} biasanya digunakan pada permainan RPG dengan \textit{sub-genre} TRPG dan JRPG.
\vspace{1ex}

Digunakannya metode $k$-NN, Distribusi Normal, dan Naive Bayes dalam melakukan proses distribusi \textit{stats} dalam permainan RPG menjadikan \textit{stats} dari pemain dan musuh menjadi terpola dari level terendah ke level tertinggi untuk pemain, mejadi tersebar dengan merata untuk distribusi musuh yang harus dihadapi oleh pemain disetiap levelnya. Hal semacam ini tentu saja sangat memudahkan pengembang permainan dalam mendesain permainan, terlebih lagi dalam pembuatan permainan dengan level yang tinggi, karakter, dan musuh yang berjumlah banyak.
\vspace{1ex}

Penggunaan \textit{Neural Network} untuk klasifikasi \textit{hero} pada permainan Dota 2 mampu mencapai 64\% \textit{hero} berhasil terklasifikasi. Sedangkan untuk klasifikasi karakter pemain yang dibuat menggunakan metode $k$-NN, Distribusi Normal, dan Naive Bayes dengan jumlah 8 karakter mampu mencapai 50\% pada data \textit{testing} dalam artian 50\% dari karakter mampu diklasifikasi tipenya. Kemudian untuk klasifikasi karakter musuh yang juga dibuat menggunakan metode tersebut dengan jumlah 400 karakter mampu mencapai 42.5\% pada data \textit{testing} dalam artian 42.5\% dari karakter mampu diklasifikasi tipenya.
\vspace{1ex}

Tinggi dan rendahnya hasil dari klasifikasi sangat bergantung pada pola \textit{stats} yang dihasilkan itu sendiri, dari pola \textit{stats} tersebut mampu membentuk koerelasi fitur yang banyak dan berpengaruh. Maksud dari berpengaruh disini adalah mampu membentuk pola \textit{stats} sesuai dengan tipenya.   

\section{Saran}
\label{sec:sec4_saran}
\vspace{1ex}

Dalam penelitian kedepannya sangat disarankan untuk meneliti tentang bagaimana mensimulasikan pertaarungan secara otomatis pada permainan genre RPG. Hal tersebut bertujuan guna mencoba mensimulasikan nilai yang dihasilkan dari penelitian ini yang berupa \textit{stats}.